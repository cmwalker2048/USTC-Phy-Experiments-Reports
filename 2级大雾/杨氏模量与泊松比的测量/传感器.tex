\documentclass[a4paper,UTF8]{ctexart}
\usepackage{threeparttable}%画图时候的脚注
\usepackage{amsmath, amsthm, amssymb, amsfonts, hyperref, mathrsfs}%美国数学学会的包+?
\usepackage{geometry} %控制界面
\usepackage{bookmark}
\usepackage{fancyhdr} % header & footer
\usepackage{tikz} %作图
\usepackage{graphicx} %插入图片的宏包
\usepackage{float} %设置图片浮动位置的宏包
\usepackage{subfigure} %插入多图时用子图显示的宏包
\usepackage{listings} %引用代码
\usepackage{physics,mathtools} %物理数学工具
\usepackage{fancyhdr}%页眉和页脚
\geometry{top=2.5cm,bottom=2.5cm,left=2.5cm,right=2.5cm} % 布局要求
\pagestyle{plain} % 角标要求
% code
\lstset{
    basicstyle          =   \sffamily,          % 基本代码风格
    keywordstyle        =   \bfseries,          % 关键字风格
    commentstyle        =   \rmfamily\itshape,  % 注释的风格,斜体
    stringstyle         =   \ttfamily,  % 字符串风格
    flexiblecolumns,                % 别问为什么,加上这个
    numbers             =   left,   % 行号的位置在左边
    showspaces          =   false,  % 是否显示空格,显示了有点乱,所以不现实了
    numberstyle         =   \zihao{-5}\ttfamily,    % 行号的样式,小五号,tt等宽字体
    showstringspaces    =   false,
    captionpos          =   t,      % 这段代码的名字所呈现的位置,t指的是top上面
    frame               =   lrtb,   % 显示边框
}

\pagestyle{fancy}
\lhead{\small 大学物理综合实验A}
\chead{}
\rhead{\small PB21000085 王潇羽}
\lfoot{}
\cfoot{第\thepage 页,共\pageref*{lastpage} 页}
\rfoot{}

\begin{document}

\begin{center}
    \textbf{\Huge 传感器系列实验}
\end{center}

\section{实验目的}
    通过电阻应变片传感器与压力传感器,让学生明白传感器的基本知识。通过搭建单桥,半桥,全桥电路,
让学生明白微小量的测量方法。并且,通过后续的数据处理,让学生学会传感器精确度的测量与运用软件拟合的方法

\section{实验原理}
    \subsection*{电阻应变片传感器}
    一根金属导线在其拉长时电阻增大,在其受压
时电阻减小,这个规律被称为金属材料得电阻应变
效应.设有一段长为$l$,截面积为$A$,电阻率为$\rho$
的固态导体,它具有的电阻为:
\begin{equation}
    R = \rho \frac{l}{A} \label{eqs:1}    
\end{equation}
当它受到轴向力$F$而被拉伸(或压缩)时,其$l,A,\rho$
均发生改变(式(\ref*{fig:1})所示),因而导体的电阻
也随之发生变化。
\begin{figure}[!hp]
    \centering
    \includegraphics[scale=0.9]{fig/范例图片/传感器_拉伸.jpg}
    \caption{导体拉伸后的参数变化}
    \label{fig:1}
\end{figure}

对式(\ref*{eqs:1})两边取对数后再微分,即可求得其电阻的相对
变化:
\begin{equation}
    \frac{dR}{R}=\frac{dl}{l}-\frac{dA}{A}+\frac{d\rho}{\rho}
    \label{eqs:2}
\end{equation}
式中,$(dl/l=\epsilon)$为材料的轴向应变,常用单位
$\mu\epsilon(1\mu\epsilon=1*10^{-6})$;而$\frac{dA}{A}
=2\frac{dr}{r}=-2\mu\epsilon$。其中,$r$为导体半径,受拉时$r$
缩小;$\mu$为导体的材料的泊松比。代入式(\ref*{eqs:2})可得:
\begin{equation}
    \frac{dR}{R}=(1+2\mu)\varepsilon+\frac{d\rho}{\rho}
    \label{eqs:3}
\end{equation}
上式中电阻率相对变化与导体材料的应变线性相关,因此导体材料的应变电阻
效应为:
\begin{equation}
    \frac{\Delta R}{R}=K\cdot\varepsilon
    \label{eqs:4}
\end{equation}
式中,$K$为应变片灵敏系数
\newpage
    电阻应变片把机械应变信号转换为$\frac{\delta R}{R}$后,由于应变量
及其应变电阻变化一般都很微小,需要测量电路将应变计的$\frac{\delta R}{R}$
变化转换成可用的电压或电路输出。直流电桥电路时常用的测量电路中的一种,
如图(\ref*{fig:2})所示。
\begin{figure}[!hp]
    \centering
    \includegraphics[scale=0.9]{fig/范例图片/传感器_单桥_半桥_全桥.jpg}
    \caption{直流电路电桥示意图}
    \label{fig:2}
\end{figure}

    图(\ref*{fig:2})$(a)$为单臂电桥电路,$R_1$为电阻应变片,其中三个桥臂连接
固定电阻$R_2,R_3和R_4$。当$R_1$未受力,电桥达到初始平衡时,但不代表电压表无示数,
有$\frac{R_1}{R_2}=\frac{R_3}{R_4}$,输出电压$U_a=0$。当应变片$R_1$受应变引起
$\delta R$电阻时,电压示数发生变化。此时,电桥输出电压$U_a$在满足$\Delta R\leq R_1$
且$\frac{R_1}{R_2}=\frac{R_3}{R_4}$的条件下为:
\begin{equation}
    U_0=\frac{1}{4}\frac{\Delta R}{R}U
    \label{eqs:5}
\end{equation}
电桥的灵敏度为:
\begin{equation}
    S=\frac{1}{4}U
    \label{eqs:6}
\end{equation}
与上讨论同理,可以得到式(\ref*{fig:2})$(b)$为半桥电路,输出电压和电桥的灵敏度分别为
\begin{equation}
    U_0=\frac{1}{2}\frac{\Delta R}{R}U
    \label{eqs:7}    
\end{equation}
\begin{equation}
    S=\frac{1}{2}U
    \label{eqs:8}
\end{equation}
式(\ref*{fig:2})$(c)$为全桥电路,输出电压和电桥的灵敏度分别为
\begin{equation}
    U_0=\frac{\Delta R}{R}U
    \label{eqs:9}    
\end{equation}
\begin{equation}
    S=U
    \label{eqs:10}
\end{equation}

    \subsection*{压阻传感器}
半导体单晶硅,锗等材料在外力作用下的电阻率将发生变化
这种现象称为压阻效应。利用压阻效应开发的传感器称为压
阻式传感器。它有两种类型:一是利用半导体材料的体电阻
,制作半导体应变计;其灵敏度要比金属应变计高2个数量级
;另一个是在半导体单晶硅(锗)的基底上利用半导体集成工艺
中的扩散技术,将弹性敏感元件与应变(转换)元件合二为一,
制成扩散硅压阻式传感器。固体材料的压阻效应可以表示为:
\begin{equation}
    \frac{\Delta R}{R} \approx \frac{\Delta\rho}{\rho}=K\sigma
    \label{eqs:11}
\end{equation}
式中,压阻系数K是表征固体材料压阻效应的特性参数,
$\sigma$是材料所承受的应力

    扩散硅压阻式传感器是将电阻条集成在单晶硅膜片上,
制成硅压阻芯片,并将此芯片的周边固定封装于外壳之内,
引出电极引线(式(\ref*{fig:3})所示)。硅膜上受到均布
压力$P$产生的应力是不均匀的,且有正应力区和负应力区。
利用这一特性,选择适当位置布置电阻,两条受拉应力的
电阻与另两条受压应力的电阻接入电桥的四臂构成差动全桥
电路,就可得到正比于力变化的电信号输出(式(\ref*{fig:4})所示),
这样既提高了输出灵敏度,又起到热补偿作用。
\begin{figure}[!hp]
    \centering
    \begin{minipage}{200pt}
        \centering
        \includegraphics[scale=1.2]{fig/范例图片/压阻传感器结构.jpg}
        \caption{压阻传感器结构}
        \label{fig:3}
    \end{minipage}
    \begin{minipage}{200pt}
        \centering
        \includegraphics[scale=1.2]{fig/范例图片/全桥测量电路.jpg}
        \caption{全桥测量电路}
        \label{fig:4}
    \end{minipage}
\end{figure}

    在恒压源供电时,当应变膜无形变时,四个应变电阻阻值相等,桥路两端输
出电压$U_0=0$。当气体进入压力强并作用于硅应变膜上时,应变膜弯曲形变,
从而使其应变电阻值发生变化。通过合理设计,可以使应变膜受到应力作用时, 
$R_1,R_4$与$R_2,R_3$有相反等量变化,电桥的输出电压可以表示为:
\begin{equation}
    U_0=\frac{\Delta R}{R}U
    \label{eqs:12}
\end{equation}
由式(\ref*{eqs:12})可见,电桥输出电压与电阻变化成正比,与电源电压成正比。

\section{数据处理}
{\centering\subsection*{电阻应变片传感器}}
\subsection*{应变力压力特性研究}
    认为压力片电阻比未受力时变大时,是应变片收到拉伸,反之,是受到
压缩力的作用,有实验数据如表(\ref*{table:1}),其中$R_1,R_2,R_3,R_4$,与
图(\ref*{fig:5})相同。

\begin{table}[!hp]
    \begin{minipage}{200pt}
        \centering
        \includegraphics[scale=1]{fig/范例图片/传感器位置.jpg}
        \caption{应力片传感位置}
        \label{fig:5}
    \end{minipage}
    \begin{minipage}{200pt}
        \vspace{20pt}
        \begin{center}
            \begin{tabular}{|c|c|c|c|c|}
                \hline
                1 & $R_1$ & $R_2$ & $R_3$ & $R_4$\\
                \hline 
                原电阻 & 1.0009 & 1.0012 & 0.9998 & 1.0015\\
                \hline
                受力后电阻 & 1.0013 & 1.0006 & 0.9994 & 1.0018\\
                \hline
            \end{tabular}
            %怎么把标题居中啊,不会
            \caption{应变力受力电阻变化特性}
            \label{table:1}
        \end{center}
    \end{minipage}        
\end{table}
那么,由表中数据,易得到$R_1:$压缩力,$R_2:$拉伸力,$R_3:$拉伸力,$R_4:$压缩力
\subsection*{三种特性的输出特性研究}
\subsubsection*{单桥电路}
    通过电桥平衡,依次增加砝码$m$,可测得单臂电桥的输出电压和砝码数量的关系,
如图(\ref*{table:2})所示。
\begin{table}[!hp]
    \begin{center}
        \begin{threeparttable}
            \caption{单桥电路电压-砝码数关系数据表}
            \begin{tabular}{|c|c|c|c|c|c|c|c|c|}
                \hline
                n\tnote{1}(砝码数) & 0 & 1 & 2 & 3 & 4 & 5 & 6 & 未知砝码\tnote{2}\\
                \hline
                U (电压) & 4.69 & 4.98 & 5.26 & 5.54 & 5.82 & 6.11 & 6.39 & 6.08\\
                \hline
            \end{tabular}
            \label{table:2}
            \begin{tablenotes}
                \footnotesize
                \item[1] $m_{砝码}=50kg$
                \item[2] 没有标注的砝码,要求用拟合曲线得出质量
            \end{tablenotes}
        \end{threeparttable}
    \end{center}
\end{table}

    可以通过origin拟合得到如下图表,曲线如图(\ref*{fig:6})所示

\begin{figure}[!hp]
    \vspace{-10pt}
    \begin{minipage}{0.65\linewidth}
        \centering
        \includegraphics[scale=0.5]{fig/单桥电路u-n关系.jpg}   
        \label{fig:6}
    \end{minipage}
    \hfill
    \begin{minipage}{0.3\linewidth}
        \begin{center}
            \begin{tabular}{c|c}
                \hline
                Model & Line\\
                \hline
                Equation & $y=A+B*x$\\
                \hline
                A(Intercept) & 4.69286\\
                \hline
                B(Slape) & 0.28286\\
                \hline
                R-Square & 0.9997\\
                \hline
            \end{tabular}
        \end{center}
    \end{minipage}
    \caption{单桥电电压-砝码数量关系图}
\end{figure}

并即得到拟合电压-砝码数关系式
\begin{equation*}
    U = U_0+\frac{\Delta U}{\Delta n}n
\end{equation*}
并将未知砝码代入拟合式,得到$m_{unknown}=245.2kg$,
灵敏度$\frac{\Delta U}{\Delta n}=B=0.28286$。
\newpage
\subsubsection*{半桥电路}
    与全桥电路类似,实验测得输出电压与砝码数量的关系,
如表(\ref*{table:3})所示。

\begin{table}[!hp]
    \begin{center}
        \begin{threeparttable}
            \caption{半桥电路电压-砝码数关系数据表}
            \begin{tabular}{|c|c|c|c|c|c|c|c|c|}
                \hline
                n\tnote{1}(砝码数) & 0 & 1 & 2 & 3 & 4 & 5 & 6 & 未知砝码\tnote{2}\\
                \hline
                U (电压) & 2.06 & 2.68 & 3.32 & 3.95 & 4.56 & 5.18 & 5.79 & 5.2\\
                \hline
            \end{tabular}
            \label{table:3}
            \begin{tablenotes}
                \footnotesize
                \item[1] $m_{砝码}=50kg$
                \item[2] 没有标注的砝码,要求用拟合曲线得出质量
            \end{tablenotes}
        \end{threeparttable}
    \end{center}
\end{table}
利用origin作图得到电压与砝码数的关系,曲线如图(\ref*{fig:7})所示。
将未知砝码代入拟合式,得到$m_{unknown}=251.7kg$,
灵敏度$\frac{\Delta U}{\Delta n}=B=0.6225$。
\begin{figure}[!hp]
    \begin{minipage}{0.65\linewidth}
        \centering
        \includegraphics[scale=0.5]{fig/半桥电路u-n关系.jpg}
        \label{fig:7}
    \end{minipage}
    \hfill
    \begin{minipage}{0.3\linewidth}
        \begin{center}
            \begin{tabular}{c|c}
                \hline
                Model & Line\\
                \hline
                Equation & $y=A+B*x$\\
                \hline
                A(Intercept) & 2.06679\\
                \hline
                B(Slape) & 0.6225\\
                \hline
                R-Square & 0.9999\\
                \hline
            \end{tabular}
        \end{center}
    \end{minipage}
    \caption{半桥电路电压-砝码数量关系图}
\end{figure}

\subsubsection*{全桥电路}
与上面两个同理,都是利用实验数据,得到电压与砝码数目的
关系,再利用origin作图得到,附上实验结果与数据处理,
不再赘述
\begin{table}[!hp]
    \begin{center}
        \begin{threeparttable}
            \caption{半桥电路电压-砝码数关系数据表}
            \begin{tabular}{|c|c|c|c|c|c|c|c|c|}
                \hline
                n\tnote{1}(砝码数) & 0 & 1 & 2 & 3 & 4 & 5 & 6 & 未知砝码\tnote{2}\\
                \hline
                U (电压) & 3.78 & 5.05 & 6.31 & 7.58 & 8.84 & 10.11 & 11.38 & 10.12\\
                \hline
            \end{tabular}
            \label{table:3}
            \begin{tablenotes}
                \footnotesize
                \item[1] $m_{砝码}=50kg$
                \item[2] 没有标注的砝码,要求用拟合曲线得出质量
            \end{tablenotes}
        \end{threeparttable}
    \end{center}
\end{table}

{\centering\subsection*{压阻传感器}}

\subsection*{检测电子元件}
    电路一切正常,测得二极管导通电压$U_{diode}=0.6059V$
三极管导通电压$U_{EB}=1.0443V,U_{BC}=1.0438V$
\subsection*{压阻传感器输出特性}
    通过搭建装置图(\ref*{fig:8}),测量得到传感器输出电压U与
正压力P关系,分别就充气与放气测得两组数据,如表(\ref*{table:4}),
和表(\ref*{table:5})所示。
\begin{figure}[!hp]
    \centering
    \includegraphics[scale=1.2]{fig/范例图片/传感器_U_P关系图.jpg}
    \caption{压力传感器输出特性测量}
    \label{fig:8}
\end{figure}
\begin{table}[!hp]
    \begin{center}
        \begin{threeparttable}
            \caption{U-P关系图(充气)}
            \begin{tabular}{|c|c|c|c|c|c|c|c|c|c|c|c|c|c|}
                \hline
                $U(V)$&0.246&0.500&0.596&0.672&0.752&1.022&1.24&1.464&1.64&1.911&2.083&2.37&2.49\\
                \hline
                $P_{extra}$\tnote{*}(MPa) &0&0.004&0.006&0.008&0.010&0.016&0.020&0.026&0.030&0.036&0.040&0.046&0.050\\
                \hline
            \end{tabular}
            \label{table:4}
            \begin{tablenotes}
                \footnotesize
                \item[*] 和外界环境的压力差
            \end{tablenotes}
        \end{threeparttable}
    \end{center}
\end{table}
\begin{table}[!hp]
    \begin{center}
        \begin{threeparttable}
            \caption{U-P关系图(放气)}
            \begin{tabular}{|c|c|c|c|c|c|c|c|c|c|c|c|c|}
                \hline
                $U(V)$&2.103&1.971&1.818&1.637&1.502&1.440&1.186&1.113&0.957&0.763&0.685&0.59\\
                \hline
                $P_{extra}$\tnote{*}(MPa) &0.040&0.038&0.034&0.030&0.028&0.026&0.020&0.018&0.016&0.010&0.008&0.006\\
                \hline
            \end{tabular}
            \label{table:5}
            \begin{tablenotes}
                \footnotesize
                \item[*] 和外界环境的压力差
            \end{tablenotes}
        \end{threeparttable}
    \end{center}
\end{table}
\newpage
当为冲气的时候,利用origin做图(\ref*{fig:9}),并求得灵敏度
$S_{inflation}=\frac{\Delta U}{\Delta P}=B=44.85(V/MPa)$
\begin{figure}[!hp]
    \vspace{-10pt}
    \begin{minipage}{0.65\linewidth}
        \centering
        \includegraphics[scale=0.5]{fig/U-P充气.jpg}   
    \end{minipage}
    \hfill
    \begin{minipage}{0.3\linewidth}
        \begin{center}
            \begin{tabular}{c|c}
                \hline
                Model & Line\\
                \hline
                Equation & $y=A+B*x$\\
                \hline
                A(Intercept) & 0.3036\\
                \hline
                B(Slape) & 44.8461\\
                \hline
                R-Square & 0.9987\\
                \hline
            \end{tabular}
        \end{center}
    \end{minipage}
    \caption{U-P关系图(充气)}
    \label{fig:9}
\end{figure}

同理,当放气的时候,得到关系如图(\ref*{fig:10}),并由灵敏度
$S_{deflation}=43.93(V/MPa)$
\begin{figure}[!hp]
    \vspace{-10pt}
    \begin{minipage}{0.65\linewidth}
        \centering
        \includegraphics[scale=0.5]{fig/U-P放气.jpg}   
    \end{minipage}
    \hfill
    \begin{minipage}{0.3\linewidth}
        \begin{center}
            \begin{tabular}{c|c}
                \hline
                Model & Line\\
                \hline
                Equation & $y=A+B*x$\\
                \hline
                A(Intercept) & 0.3106\\
                \hline
                B(Slape) & 43.9347\\
                \hline
                R-Square & 0.9971\\
                \hline
            \end{tabular}
        \end{center}
    \end{minipage}
    \caption{U-P关系图(放气)}
    \label{fig:10}
\end{figure}
\newpage
对上述充气与放气得到的灵敏度取平均得到,$\bar{S}=\frac{S_{inflation}}{S_{deflation}}
=44.39(V/Mpa)$,
相较于对灵敏度取平均得到解,更好的做法应该是将放气与充气的点
一起用origin拟合,直接得到新的拟合曲线,如图({fig:12})所示。
并直接得到$\bar{S}=43.93(V/MPa)$,与
\begin{equation}
    U=U_0+\frac{\Delta U}{\Delta P}P=0.3106+43.93P
    \label{eqs:13}
\end{equation}
\begin{figure}[!hp]
    \begin{minipage}{0.65\linewidth}
        \centering
        \includegraphics[scale=0.5]{fig/U-P.jpg}   
    \end{minipage}
    \hfill
    \begin{minipage}{0.3\linewidth}
        \begin{center}
            \begin{tabular}{c|c}
                \hline
                Model & Line\\
                \hline
                Equation & $y=A+B*x$\\
                \hline
                A(Intercept) & 0.30492\\
                \hline
                B(Slape) & 44.46286\\
                \hline
                R-Square & 0.9980\\
                \hline
            \end{tabular}
        \end{center}
    \end{minipage}
    \caption{U-P关系图}
    \label{fig:12}
\end{figure}

\subsection*{压阻传感器报警系统}
    搭建如图(\ref*{fig:13})所示的电路图,并逐渐增加气压,
记录下蜂鸣器刚开始报警时候的气压$P=0.006MPa$,代入公式(\ref*{eqs:13})
得$U=0.5742V$。
\begin{figure}[!hp]
    \centering
    \includegraphics[scale=1]{fig/范例图片/传感器_警报器.jpg}
    \label{fig:13}
\end{figure}


\newpage
\section{实验结果分析}
\subsection*{应变力压力传感器}
\begin{itemize}
    \item 结果线性度非常好,并且精确度满足理论中的
            单桥,半桥,全桥的1:2:4倍数关系
    \item 但不同电路测出的未知砝码质量由一定误差,
            可能是导线电阻或接口处的电阻影响
\end{itemize}
\subsection*{压阻传感器}
    \begin{itemize}
        \item 结果满足线性的规律,并且充气和放气时候的灵敏度相差不大
        \item 但警报器读出的导通电压和直接测量的导通电压相差极大,可能原因如下
        \begin{enumerate}
            \item 警报时候气压太小,气压计测量不精确
            \item 产生警报速度太快,难以快速反应,导致误差
            \item 直接测量三极管时测错了(但测量时测过好几遍,多次结果相差不大)
        \end{enumerate}
    \end{itemize}

\section{思考题}
\subsection*{应变片压力传感器}
    \begin{enumerate}
        \item 注意如下
        \begin{itemize}
            \item 应变片拉伸与收缩方向要与电路图中一致,不然无法正常工作
            \item 由于电路比较复杂,不要发生断路,短路等基本电路故障
            \item 尽量少的调用多个导线的接口,因为接口处的电阻相比电线内电阻相当大,容易引起较大误差
        \end{itemize}
        \item 基本相同,详细分析请见上文中的"实验结果分析"
    \end{enumerate}
\subsection*{压阻传感器}
    \begin{enumerate}
        \item 压阻效应就是半导体板电阻受其承受的应力变化而变化,引起电阻变化从而导致电路中的性质发生变化,满足公式
        \begin{equation*}
            \frac{\Delta R}{R}\approx\frac{\Delta \rho}{\rho}=K\sigma
        \end{equation*}
            $\sigma$为材料承受的应力
        \item 其异同如下
        \begin{itemize}
            \item 压阻传感器与其上收到的压力成一定关系
            \item 电阻应变片与其伸长还是压缩成一定关系
        \end{itemize}
            两者都与其受力情况有关,但引起电阻改变的原因不同
    \end{enumerate}

\label{lastpage}
\end{document}