\documentclass[a4paper,UTF8]{ctexart}

\usepackage{amsmath, amsthm, amssymb, amsfonts, hyperref, mathrsfs}%美国数学学会的包+?
\usepackage{geometry} %控制界面
\usepackage{bookmark}
\usepackage{fancyhdr} % header & footer
\usepackage{appendix} % 附录
\usepackage{tikz} %作图
\usepackage{graphicx} %插入图片的宏包
\usepackage{float} %设置图片浮动位置的宏包
%\usepackage{subfigure} %插入多图时用子图显示的宏包
\usepackage{listings} %引用代码
\usepackage{physics,mathtools} %物理数学工具
\usepackage{comment}
\usepackage{framed}
\usepackage{caption}
\usepackage{subcaption}
\geometry{top=2.5cm,bottom=2.5cm,left=2.5cm,right=2.5cm} % 布局要求
\pagestyle{fancy} % fancy分格
\fancyhf{} % 清除所有页眉页脚
\renewcommand\headrulewidth{0.6pt}
\renewcommand\footrulewidth{0.6pt}
% font
\setCJKmainfont{Noto Serif CJK SC}[BoldFont={Noto Serif CJK SC Bold}, ItalicFont=]
\lhead{何金铭 PB21020660$\mid$座位号:8}
\cfoot{Quantum Key Distribution 实验报告}
\rhead{\thepage}
\lfoot{2024.4.23}
\rfoot{USTC}
%\bibliographystyle{plain} % 引用样式
\everymath{\displaystyle} % display
%============================================================

\begin{document}

\begin{center}
    \textbf{\Large Quantum Key Distribution 实验报告}
    \par \text{\large 何金铭 PB21020660}
\end{center}

实验目的,实验原理,实验内容已于预习报告中给出,这里不再赘述。

\section{实验结果与分析}

\subsection{基矢对比}

在完成:

\begin{enumerate}
    \item 同步设置
    \item 利用MPC进行偏振调节
\end{enumerate}

等操作后,我们进行基矢对比得错误率为$4\%$,小于错误率要求$5\%$

\subsection{量子密钥应用演示}

\subsubsection{聊天加密}

Bob传输一串字母"Hello! How are you? Nice to meet you!",在接收端ALice解密后得到原文;若不解密,则接受到的是乱码。

\subsubsection{图片加密}

Bob传输一张图片,在接收端Alice解密后得到原图;若不解密,则接受到的是带有很多噪声的图片。

\section{实验结论}

\begin{enumerate}
    \item 同步对于实验结果的影响很大,若未同步,则会使得光子计数率低,导致错误率高,几乎不能通信;
    \item 通过了这个实验,我们了解了QKD的基本工作原理,并且利用了QKD进行了几个简单的加密通信;
\end{enumerate}

\section{思考题}

\subsection{量子保密通信为什么是无条件安全的,其物理基础是什么?}

其物理基础是量子态不可克隆定理,即未知量子态不能精确克隆(对任意输入态)。

\subsection{量子不可克隆定理是什么?}

未知量子态不能精确克隆(对任意输入态)

下面证明这个定理:

$|\phi\rangle$ 和 $|\psi\rangle$ 是两个任意的量子状态,我们要把这两个状态拷贝到另一个与他们完全无关的状态$|k\rangle$上。我
们用一个幺正算符$U$来描述这个过程。则这个拷贝算符必须具备以下性质:
$$\begin{array}{l}U(|\phi\rangle\otimes|k\rangle)=|\phi\rangle\otimes|\phi\rangle\\U(|\psi\rangle\otimes|k\rangle)=|\psi\rangle\otimes|\psi\rangle\end{array}$$
内积$\langle U(\phi\otimes k)|U(\psi\otimes k)\rangle$可得出以下两个等式:
$$\begin{array}{l}\langle U(\phi\otimes k)|U(\psi\otimes k)\rangle=\langle\phi\otimes\phi|\psi\otimes\psi\rangle\\\langle U(\phi\otimes k)|U(\psi\otimes k)\rangle=\langle\phi\otimes k|\psi\otimes k\rangle\end{array}$$
这样便得到了:
$$\begin{aligned}&\langle\phi\otimes\phi|\psi\otimes\psi\rangle=\langle\phi\otimes k|\psi\otimes k\rangle,\\&\to\\&\langle\phi|\psi\rangle\langle\phi|\psi\rangle=\langle\phi|\psi\rangle\langle k|k\rangle\:.\end{aligned}$$
因为$\langle k|k\rangle=1$,所以得出
$$\langle\phi|\psi\rangle^{2}=\langle\phi|\psi\rangle.$$

这个等式仅有的两个解是$\langle\phi|\psi\rangle=0$ 和 $\langle\phi|\psi\rangle=1$。这意味着,要么$\phi=\psi$ (当$\langle\phi|\psi\rangle=1$),要么 $\phi$ 与$\psi$ 正交 (当$\langle\phi|\psi\rangle=0$)。只能
够克降相同或正交的状态,这并不是我们最初假设的任意状态的完全克隆,不可克隆原理证明完毕。

\subsection{实验中所使用的光源是单光子源还是其它什么光源?}

实际情况中,我们不使用单光子源,而是使用弱相干光源,因为:

\begin{enumerate}
    \item 单光子源实现较困难;
    \item 信道损耗大;
\end{enumerate}

\subsection{使用非单光子源可能会有什么问题,有没有办法消除?(调研,选做)}

解决方法: Decoy State (诱骗态方案)

由于Eve只分裂发送多光子脉冲,分裂导致多光子脉冲的损耗特
性发生改变,一般表现为损耗降低;
原因:Eve为了保证多光子脉冲尽可能的被Bob接受到,从而让
Bob以为是有用传输,用以生成密钥,这样Eve等AB公开测量基
信息后就可以提取密钥;
我们可以发送一个主要以多光子脉冲分布的光场,如果Eve用这
种方式攻击,则我们会看到多光子脉冲的损耗会降低!从而探测
到Eve的存在;

Decoy State QKD 协议如下:

\begin{enumerate}
    \item Alice 随机发送信号态或诱骗态给Bob;
    \item  Bob公开每一次发送信号时候接收到;
    \item Alice公开说明具体每一次发送的是信号态还是诱骗态;
    \item Alice 和 Bob 计算信号态和诱骗态各自传输成功的概率;
如果Eve只选择的发送两光子态,则A、B会发现诱骗态B)
的接受成功概率非常高,从而确认Eve的存在;没有Eve时
,获取密钥的方法如BB84同。
\end{enumerate}

\subsection{如何降低实验中的错误率?}

\begin{enumerate}
    \item 提升激光器的单光子性
    \item 降低光纤中的光子损耗
    \item 提高单光子探测器的探测效率
    \item 提升偏振度的对比度
\end{enumerate}

\subsection{请说说实验中调节MPC起到的作用是什么,如何判断调节好了,为什么这样判断?}

作用为调节光子的偏振态$\ket{H},\ket{V},\ket{+},\ket{-}$

判断的标准是偏振对比度$\ket{H}:\ket{V}, \ket{+}:\ket{-}$大于20:1或小于1:20,因为说明两者强度比很大,可近似为理想线性偏
振光了。

\end{document}